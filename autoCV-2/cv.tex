%-----------------------------------------------------------------------------------------------------------------------------------------------%
%	The MIT License (MIT)
%
%	Copyright (c) 2021 Jitin Nair
%
%	Permission is hereby granted, free of charge, to any person obtaining a copy
%	of this software and associated documentation files (the "Software"), to deal
%	in the Software without restriction, including without limitation the rights
%	to use, copy, modify, merge, publish, distribute, sublicense, and/or sell
%	copies of the Software, and to permit persons to whom the Software is
%	furnished to do so, subject to the following conditions:
%	
%	THE SOFTWARE IS PROVIDED "AS IS", WITHOUT WARRANTY OF ANY KIND, EXPRESS OR
%	IMPLIED, INCLUDING BUT NOT LIMITED TO THE WARRANTIES OF MERCHANTABILITY,
%	FITNESS FOR A PARTICULAR PURPOSE AND NONINFRINGEMENT. IN NO EVENT SHALL THE
%	AUTHORS OR COPYRIGHT HOLDERS BE LIABLE FOR ANY CLAIM, DAMAGES OR OTHER
%	LIABILITY, WHETHER IN AN ACTION OF CONTRACT, TORT OR OTHERWISE, ARISING FROM,
%	OUT OF OR IN CONNECTION WITH THE SOFTWARE OR THE USE OR OTHER DEALINGS IN
%	THE SOFTWARE.
%	
%
%-----------------------------------------------------------------------------------------------------------------------------------------------%

%----------------------------------------------------------------------------------------
%	DOCUMENT DEFINITION
%----------------------------------------------------------------------------------------

% article class because we want to fully customize the page and not use a cv template
\documentclass[a4paper,12pt]{article}

%----------------------------------------------------------------------------------------
%	FONT
%----------------------------------------------------------------------------------------

% % fontspec allows you to use TTF/OTF fonts directly
% \usepackage{fontspec}
% \defaultfontfeatures{Ligatures=TeX}

% % modified for ShareLaTeX use
% \setmainfont[
% SmallCapsFont = Fontin-SmallCaps.otf,
% BoldFont = Fontin-Bold.otf,
% ItalicFont = Fontin-Italic.otf
% ]
% {Fontin.otf}

%----------------------------------------------------------------------------------------
%	PACKAGES
%----------------------------------------------------------------------------------------
\usepackage{url}
\usepackage{parskip} 	

%other packages for formatting
\RequirePackage{color}
\RequirePackage{graphicx}
\usepackage[usenames,dvipsnames]{xcolor}
\usepackage[scale=0.9]{geometry}

%tabularx environment
\usepackage{tabularx}

%for lists within experience section
\usepackage{enumitem}

% centered version of 'X' col. type
\newcolumntype{C}{>{\centering\arraybackslash}X} 

%to prevent spillover of tabular into next pages
\usepackage{supertabular}
\usepackage{tabularx}
\newlength{\fullcollw}
\setlength{\fullcollw}{0.47\textwidth}

%custom \section
\usepackage{titlesec}				
\usepackage{multicol}
\usepackage{multirow}

%CV Sections inspired by: 
%http://stefano.italians.nl/archives/26
\titleformat{\section}{\Large\scshape\raggedright}{}{0em}{}[\titlerule]
\titlespacing{\section}{0pt}{10pt}{10pt}

%for publications
\usepackage[style=authoryear,sorting=ynt, maxbibnames=2]{biblatex}

%Setup hyperref package, and colours for links
\usepackage[unicode, draft=false]{hyperref}
\definecolor{linkcolour}{rgb}{0,0.2,0.6}
\hypersetup{colorlinks,breaklinks,urlcolor=linkcolour,linkcolor=linkcolour}
\addbibresource{citations.bib}
\setlength\bibitemsep{1em}

%for social icons
\usepackage{fontawesome5}

%debug page outer frames
%\usepackage{showframe}


% job listing environments
\newenvironment{jobshort}[2]
    {
    \begin{tabularx}{\linewidth}{@{}l X r@{}}
    \textbf{#1} & \hfill &  #2 \\[3.75pt]
    \end{tabularx}
    }
    {
    }

\newenvironment{joblong}[2]
    {
    \begin{tabularx}{\linewidth}{@{}l X r@{}}
    \textbf{#1} & \hfill &  #2 \\[3.75pt]
    \end{tabularx}
    \begin{minipage}[t]{\linewidth}
    \begin{itemize}[nosep,after=\strut, leftmargin=1em, itemsep=3pt,label=--]
    }
    {
    \end{itemize}
    \end{minipage}    
    }



%----------------------------------------------------------------------------------------
%	BEGIN DOCUMENT
%----------------------------------------------------------------------------------------
\begin{document}

% non-numbered pages
\pagestyle{empty} 

%----------------------------------------------------------------------------------------
%	TITLE
%----------------------------------------------------------------------------------------

% \begin{tabularx}{\linewidth}{ @{}X X@{} }
% \huge{Your Name}\vspace{2pt} & \hfill \emoji{incoming-envelope} email@email.com \\
% \raisebox{-0.05\height}\faGithub\ username \ | \
% \raisebox{-0.00\height}\faLinkedin\ username \ | \ \raisebox{-0.05\height}\faGlobe \ mysite.com  & \hfill \emoji{calling} number
% \end{tabularx}

\begin{tabularx}{\linewidth}{@{} C @{}}
\Huge{Abderrahmane Er-raqabi} \\[7.5pt]
\href{https://github.com/AbderrahmaneErraqabi}{\raisebox{-0.05\height}\faGithub\ AbderrahmaneErraqabi} \ $|$ \ 
\href{https://www.linkedin.com/in/abderrahmane-er-raqabi-7381b0354/}{\raisebox{-0.05\height}\faLinkedin\ Abderrahmane Er-Raqabi} \ $|$ \ 
\href{https://abderrahmaneer-raqabi.com}{\raisebox{-0.05\height}\faGlobe \ abderrahmaneer-raqabi.com} \ $|$ \\ [3pt]
\href{mailto:abderrahmane.erraqabi@gmail.com}{\raisebox{-0.05\height}\faEnvelope \ abderrahmane.erraqabi@gmail.com} \ $|$ \ 
\href{tel:+000000000000}{\raisebox{-0.05\height}\faMobile \ 514-806-3430} \\
\end{tabularx}

%----------------------------------------------------------------------------------------
\section{Éducation}
\begin{tabularx}{\linewidth}{@{}l X@{}}	

2025 -- Présent & 
\textbf{Baccalauréat en génie électrique} — Polytechnique Montréal (Montréal, QC) \\
& \textit{Spécialisation en systèmes embarqués et conception électronique. Participation à l’équipe Esteban (voiture solaire).} \\[0.6em]

2023 -- 2025 & 
\textbf{DEC en Sciences de la nature} — Collège Bois-de-Boulogne (Montréal, QC) \\
& \textit{Profil scientifique axé sur la physique, la chimie et les mathématiques avancées.} \\[0.6em]

2018 -- 2023 & 
\textbf{Diplôme d’études secondaires} — École secondaire Curé-Antoine-Labelle (Laval, QC) \\
& \textit{Concentration en sciences et mathématiques; implication dans des projets technologiques et parascolaires.} \\

\end{tabularx}

%----------------------------------------------------------------------------------------

%----------------------------------------------------------------------------------------
\section{Compétences techniques}

\begin{tabularx}{\linewidth}{@{}l X@{}}

\textbf{Langages} & 
C/C++, Python, Rust, MATLAB, JavaScript, TypeScript, HTML/CSS, SQL \\[1.5em]

\textbf{Outils et plateformes} & 
KiCad, AutoCAD Electrical, Git/GitHub, STM32, Arduino, Raspberry Pi, Visual Studio Code, Linux \\[1.5em]

\textbf{Frameworks et technologies} & 
React, Next.js, TailwindCSS, Node.js, Express, Django, Flask \\[1.5em]

\textbf{Logiciels d’ingénierie} & 
LTSpice, MATLAB/Simulink, KiCad, AutoCAD Electrical, Fusion 360 \\[1.5em]

\textbf{Compétences techniques clés} & 
Conception et simulation de circuits, conception de PCB, intégration de systèmes embarqués, développement logiciel, contrôle embarqué \\[1.5em]

\vspace{1.2em}


\textbf{Domaines d’intérêt} & 
Systèmes embarqués, électronique de puissance, énergies renouvelables, robotique, instrumentation et contrôle, développement web interactif \\[1.5em]

\end{tabularx}
%----------------------------------------------------------------------------------------

%----------------------------------------------------------------------------------------
\section{Projets}

%------------------------------------------------
\begin{tabularx}{\linewidth}{@{}X r@{}}
\textbf{Système d’automatisation de comptage de comprimés} | Arduino, Logisim, Logic Design & \textit{2025} \\
\end{tabularx}
\begin{itemize}[leftmargin=1em]
    \item Conçu un système logique combinatoire et séquentiel utilisant des bascules et des décodeurs à 7 segments pour automatiser le comptage de comprimés en production.
    \item Réalisé le prototypage sur Arduino pour valider la chaîne de contrôle, incluant la synchronisation entre capteur et affichage numérique.
    \item Documenté la logique matérielle et testé les états du circuit avec des simulations et des entrées séquentielles.
\end{itemize}

\vspace{0.8em}

%------------------------------------------------
\begin{tabularx}{\linewidth}{@{}X r@{}}
\textbf{Contrôleur d’éclairage à base de STM32} | STM32, PWM, Capteurs de luminosité, C & \textit{2025} \\
\end{tabularx}
\begin{itemize}[leftmargin=1em]
    \item Développé un contrôleur intelligent pour réguler un système d’éclairage à intensité variable via microcontrôleur STM32.
    \item Implémenté le contrôle PWM, la lecture analogique des capteurs de luminosité et la communication série UART pour la configuration dynamique.
    \item Optimisé le code C embarqué pour une réponse en temps réel et une consommation minimale.
\end{itemize}

\vspace{0.8em}

%------------------------------------------------
\begin{tabularx}{\linewidth}{@{}X r@{}}
\textbf{Image-Processor — Traitement d’images RVB} | C++, Algorithmes Pixelaires & \textit{2025} \\
\end{tabularx}
\begin{itemize}[leftmargin=1em]
    \item Développé un programme C++ pour la manipulation d’images RVB (inversion, filtrage, mise à l’échelle) à l’aide d’opérations pixelaires.
    \item Implémenté des routines optimisées de lecture et d’écriture de fichiers BMP pour un traitement rapide et stable.
    \item \href{https://github.com/AbderrahmaneErraqabi/Image-processor}{Lien du démo — GitHub}
\end{itemize}

\vspace{0.8em}

%------------------------------------------------
\begin{tabularx}{\linewidth}{@{}X r@{}}
\textbf{Mini-SPICE — Solveur de circuits CC} | C++, Algèbre linéaire, Simulation de circuits & \textit{2025} \\
\end{tabularx}
\begin{itemize}[leftmargin=1em]
    \item Conçu un solveur de circuits électriques inspiré de SPICE en C++, appliquant l’analyse nodale modifiée (MNA) pour simuler des réseaux CC.
    \item Utilisé des calculs matriciels pour résoudre des systèmes d’équations et générer des rapports de tension et de courant.
    \item \href{https://github.com/AbderrahmaneErraqabi/mini-spice-dc}{Lien du démo — GitHub}
\end{itemize}

\vspace{0.8em}

%------------------------------------------------
\begin{tabularx}{\linewidth}{@{}X r@{}}
\textbf{Recherche du chemin le plus court — Modélisation par matrices d’adjacence} | SageMath, Théorie des graphes, Algèbre linéaire & \textit{2024} \\
\end{tabularx}
\begin{itemize}[leftmargin=1em]
    \item Développé un modèle mathématique et informatique pour déterminer le chemin le plus court et le plus sécuritaire dans un graphe.
    \item Implémenté et comparé les algorithmes de Dijkstra et de Floyd-Warshall sous SageMath pour divers jeux de données.
    \item Analysé les performances selon la densité des graphes et la pondération des arêtes.
\end{itemize}
\vspace{1.2em}

%----------------------------------------------------------------------------------------

%Experience
%----------------------------------------------------------------------------------------
\section{Expérience de travail}


\begin{jobshort}{\textbf{Équipe câblage} — Groupe Esteban, Polytechnique Montréal}{Août 2025 -- Présent}
Conception et implantation du réseau électrique de la voiture solaire \textit{Esteban}, incluant le routage des faisceaux, le câblage haute et basse tension, et l’intégration des sous-systèmes de contrôle et de sécurité. Collaboration directe avec l’équipe électronique pour assurer fiabilité et conformité technique. 
\end{jobshort}

\begin{jobshort}{\textbf{Gestionnaire des ventes} — SoftMoc (Montréal, QC)}{Juin 2023 -- Présent}
Encadrement d’équipe et suivi des indicateurs de performance pour optimiser les ventes et l’expérience client. 
Mise en œuvre d’initiatives ciblées de fidélisation et gestion des opérations dans un environnement à haut volume. 
\end{jobshort}

\begin{jobshort}{\textbf{Troisième clé} — Bopied (Laval, QC)}{Juin 2021 -- Mai 2023}
Responsable du service client, des commandes et de la gestion des stocks. 
Contribution à la formation des nouveaux employés et au maintien d’un environnement commercial efficace et accueillant. 
\end{jobshort}
\vspace{1.2em}


%----------------------------------------------------------------------------------------


%----------------------------------------------------------------------------------------
\section{Certifications}

\begin{tabularx}{\linewidth}{@{}p{4.3cm} X@{}}

\textbf{SIMDUT} &
Formation sur la sécurité au travail et la manipulation sécuritaire des produits chimiques — \textit{obtenue en 2025}. \\[0.4em]

\textbf{RCR et     premiers soins } &
Certification en techniques de premiers secours, incluant la réanimation cardiorespiratoire et la gestion des urgences — \textit{obtenue en 2025}. \\

\end{tabularx}
%----------------------------------------------------------------------------------------

\vfill
\end{document}
